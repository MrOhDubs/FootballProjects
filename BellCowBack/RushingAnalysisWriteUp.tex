\documentclass[]{article}
\usepackage{lmodern}
\usepackage{amssymb,amsmath}
\usepackage{ifxetex,ifluatex}
\usepackage{fixltx2e} % provides \textsubscript
\ifnum 0\ifxetex 1\fi\ifluatex 1\fi=0 % if pdftex
  \usepackage[T1]{fontenc}
  \usepackage[utf8]{inputenc}
\else % if luatex or xelatex
  \ifxetex
    \usepackage{mathspec}
  \else
    \usepackage{fontspec}
  \fi
  \defaultfontfeatures{Ligatures=TeX,Scale=MatchLowercase}
\fi
% use upquote if available, for straight quotes in verbatim environments
\IfFileExists{upquote.sty}{\usepackage{upquote}}{}
% use microtype if available
\IfFileExists{microtype.sty}{%
\usepackage{microtype}
\UseMicrotypeSet[protrusion]{basicmath} % disable protrusion for tt fonts
}{}
\usepackage[margin=1in]{geometry}
\usepackage{hyperref}
\hypersetup{unicode=true,
            pdftitle={Bring Back the Bell Cow Back?},
            pdfauthor={Orrin Wheeler},
            pdfborder={0 0 0},
            breaklinks=true}
\urlstyle{same}  % don't use monospace font for urls
\usepackage{graphicx,grffile}
\makeatletter
\def\maxwidth{\ifdim\Gin@nat@width>\linewidth\linewidth\else\Gin@nat@width\fi}
\def\maxheight{\ifdim\Gin@nat@height>\textheight\textheight\else\Gin@nat@height\fi}
\makeatother
% Scale images if necessary, so that they will not overflow the page
% margins by default, and it is still possible to overwrite the defaults
% using explicit options in \includegraphics[width, height, ...]{}
\setkeys{Gin}{width=\maxwidth,height=\maxheight,keepaspectratio}
\IfFileExists{parskip.sty}{%
\usepackage{parskip}
}{% else
\setlength{\parindent}{0pt}
\setlength{\parskip}{6pt plus 2pt minus 1pt}
}
\setlength{\emergencystretch}{3em}  % prevent overfull lines
\providecommand{\tightlist}{%
  \setlength{\itemsep}{0pt}\setlength{\parskip}{0pt}}
\setcounter{secnumdepth}{0}
% Redefines (sub)paragraphs to behave more like sections
\ifx\paragraph\undefined\else
\let\oldparagraph\paragraph
\renewcommand{\paragraph}[1]{\oldparagraph{#1}\mbox{}}
\fi
\ifx\subparagraph\undefined\else
\let\oldsubparagraph\subparagraph
\renewcommand{\subparagraph}[1]{\oldsubparagraph{#1}\mbox{}}
\fi

%%% Use protect on footnotes to avoid problems with footnotes in titles
\let\rmarkdownfootnote\footnote%
\def\footnote{\protect\rmarkdownfootnote}

%%% Change title format to be more compact
\usepackage{titling}

% Create subtitle command for use in maketitle
\newcommand{\subtitle}[1]{
  \posttitle{
    \begin{center}\large#1\end{center}
    }
}

\setlength{\droptitle}{-2em}

  \title{Bring Back the Bell Cow Back?}
    \pretitle{\vspace{\droptitle}\centering\huge}
  \posttitle{\par}
  \subtitle{Analysis of the Bell Cow back on team wins in today's NFL}
  \author{Orrin Wheeler}
    \preauthor{\centering\large\emph}
  \postauthor{\par}
      \predate{\centering\large\emph}
  \postdate{\par}
    \date{5/6/2019}

\usepackage{booktabs}
\usepackage{longtable}
\usepackage{array}
\usepackage{multirow}
\usepackage{wrapfig}
\usepackage{float}
\usepackage{colortbl}
\usepackage{pdflscape}
\usepackage{tabu}
\usepackage{threeparttable}
\usepackage{threeparttablex}
\usepackage[normalem]{ulem}
\usepackage{makecell}
\usepackage{xcolor}

\begin{document}
\maketitle

In today's NFL, there is a lot of discussion about the devaluation of
the running back position and the death of the feature back. As more and
more teams move to having a running back by committee approach, running
backs like Todd Gurley and Zeke Elliott stand as the last monuments to
the old school feature backs that take nearly all of the rushing
attempts every game.

In this analysis, we will take a look at what effect having a feature
back on your team may have. The assumption is that having one of these
great feature backs has a big positive effect on your team's rushing
statistics which will hopefully translate to wins.

We will take a look at data scraped from
\href{https://www.pro-football-reference.com}{pro-football-reference.com}
for the 2014 through 2018 regular seasons. We will take a look at the
leading rushers on each team for each year and see what sort of effects
having a feature back may have on overall team success.

Note: If a player was traded mid-season and played for multiple teams in
a given year, they can hardly be counted as a feature back so we will
ignore these players and move to the next leading rushing on a team.

For our reference, we will classify a leading rusher as a feature back
if they took more than 65\% of the attempted rushes for their team in a
given year.

By this definition, we have 19 feature backs out of a possible 160
between 2014 and 2018. Included in this list are the usual suspects of
Elliott, Gurley, Peterson, Barkley and Bell among others.

\begin{tabu} to \linewidth {>{\raggedright}X>{\raggedright}X>{\raggedleft}X>{\raggedright}X>{\raggedright}X>{\raggedleft}X>{\raggedright}X>{\raggedright}X>{\raggedright}X}
\hline
Player & Team & Year & Player & Team & Year & Player & Team & Year\\
\hline
David Johnson & Arizona Cardinals & 2016 & Ezekiel Elliott & Dallas Cowboys & 2018 & Latavius Murray & Oakland Raiders & 2015\\
\hline
David Johnson & Arizona Cardinals & 2018 & Frank Gore & Indianapolis Colts & 2015 & LeSean McCoy & Philadelphia Eagles & 2014\\
\hline
Matt Forte & Chicago Bears & 2014 & Kareem Hunt & Kansas City Chiefs & 2017 & Le'Veon Bell & Pittsburgh Steelers & 2014\\
\hline
Jordan Howard & Chicago Bears & 2016 & Melvin Gordon & Los Angeles Chargers & 2017 & Le'Veon Bell & Pittsburgh Steelers & 2017\\
\hline
Jordan Howard & Chicago Bears & 2017 & Todd Gurley & Los Angeles Rams & 2016 & Alfred Morris & Washington Redskins & 2014\\
\hline
Joe Mixon & Cincinnati Bengals & 2018 & Adrian Peterson & Minnesota Vikings & 2015 & NA & NA & \\
\hline
DeMarco Murray & Dallas Cowboys & 2014 & Saquon Barkley & New York Giants & 2018 & NA & NA & \\
\hline
\end{tabu}

\hypertarget{rushing-by-year}{%
\subsection{Rushing by year}\label{rushing-by-year}}

For the past several seasons, we have seen a sharp increase in the
importance of the passing game with the recent rules adjustments
generally favoring the offense. That should generally coincide with the
diminishing importance of the rushing game which may lead to lower
rushing stats. Lets take a look!

\includegraphics{RushingAnalysisWriteUp_files/figure-latex/unnamed-chunk-4-1.pdf}

As we can see here there is actually very little change year over year
for rushing yards. If anything, rushing yardage is actually slightly
increasing , however the relationship of the two is insignificant. This
goes against the idea that the modern NFL running game is on the
decline. This begs the question however, does an increase in team
rushing yards actually lead to increased success? Perhaps there are
simply teams that run more than others but that has no impact on the
general win percentage of teams.

\hypertarget{rushing-by-team}{%
\subsection{Rushing by Team}\label{rushing-by-team}}

First lets get a sense of which teams tend to run more. Teams like the
Patriots, Steelers, and Seahawks are perennial playoff teams but do
these teams tend to rush for more yards than other teams?

To get a sense of which teams to look out for, lets take a look at the
teams with the highest average win total. Generally 10 wins will get a
team to a playoff spot so which teams have been averaging over 10 wins a
year?

Teams like the Patriots and Steelers are always in the playoffs but is
this a result of a more effective running game?

\begin{table}[H]
\centering
\begin{tabular}{l|r}
\hline
Team & Average Wins\\
\hline
New England Patriots & 12.4\\
\hline
Kansas City Chiefs & 10.8\\
\hline
Pittsburgh Steelers & 10.8\\
\hline
Los Angeles Chargers & 10.5\\
\hline
Seattle Seahawks & 10.2\\
\hline
\end{tabular}
\end{table}

\includegraphics{RushingAnalysisWriteUp_files/figure-latex/unnamed-chunk-7-1.pdf}

Of the teams averaging over 10 wins per season the past 5 years, only
the Seahawks are averaging more than 2000 yards rushing over that same
time period. The Patriots, Steelers, and Chiefs all fall in the middle
of the pack while the Chargers have been averaging one of the lowest
rushing totals in the entire NFL over the past 5 years.

\hypertarget{wins-by-team-rushing}{%
\subsection{Wins by Team Rushing}\label{wins-by-team-rushing}}

All of this doesn't look too good on our assumption that more yards from
a team's bell cow back would result in more wins. If we compare a team's
total rushing yards (across all rushing attempts) though we start to see
a slightly different story.

\includegraphics{RushingAnalysisWriteUp_files/figure-latex/unnamed-chunk-9-1.pdf}

Here we can see positive correlation between wins and total team rushing
yards. The strength is still fairly weak (R\textsuperscript{2} =
0.1419), but at this sample size, the correlation is statistically
significant at the p \textless{} .01 level. With a slope of .004 we can
estimate for for every additional 250 yards rushed, a team can expect an
additional win. However, these are still total team rushing yards. The
question we are asking is whether or not we see a significant increase
in wins when a team utilizes a feature running back. So now let's dive
down into player specific statistics.

\hypertarget{wins-by-player-rushing}{%
\subsection{Wins by player rushing}\label{wins-by-player-rushing}}

First lets take a look simply at the number of wins as a function of how
many yards the leading rusher gained.

\includegraphics{RushingAnalysisWriteUp_files/figure-latex/unnamed-chunk-11-1.pdf}

As we expect, we see a positive trend when estimating the expected
number of wins by the number of rushing yards by the leading rusher on a
team. Here for roughly every 355 yards gained by the leading rusher a
given team can expect an additional win. This relationship is
significant as well at the p \textless{} .05 level as well. So comparing
to total team yards, we see a weaker effect but one that is still
statistically significant, albeit at a lower confidence level.

We need to be careful however because team rushing yards and leading
rusher yards are highly correlated with a Pearson correlation
coefficient of 0.568. Since these two fields are highly correlated we
expect them to encode much of the same information. What we really care
about for this analysis is does giving the featured running back a
higher percentage of the workload result in more wins. To do this, we
need to look at wins as a function of the percent of total team rushing
yards that were gained by the leading rusher. Similarly we want to look
at wins as a function of the percent of team rushing attempts taken by
the leading rusher.

\hypertarget{wins-by-player-rush-percentage}{%
\subsection{Wins by Player Rush
Percentage}\label{wins-by-player-rush-percentage}}

As evidenced below, when you look at wins as a function of what
percentage of yards the leading rusher gained for a team, we see a much
weaker trend.

\includegraphics{RushingAnalysisWriteUp_files/figure-latex/unnamed-chunk-13-1.pdf}

The trend is still positive, but with an extremely weak
R\textsuperscript{2} = 0.0016. Looking further, we can see that this
relationship is in no way significant with a p-value of 0.61. This
distribution can be produced by and potentially is a result of
randomness. Things look even more bleak when we compare wins to the
player attempts percentage.

\hypertarget{wins-by-player-attempt-percentage}{%
\subsection{Wins by Player Attempt
Percentage}\label{wins-by-player-attempt-percentage}}

Recall from before that being a feature back is defined as having 65\%
of your team's rushing attempts or greater so here we see a clear
segregation of feature and non-feature backs.

\includegraphics{RushingAnalysisWriteUp_files/figure-latex/unnamed-chunk-15-1.pdf}

Here we actually even see a slightly negative slope to our line
indicating that one player taking a larger portion of a teams snaps may
actually be detrimental to a teams success. This trend is very slight
however and contains no statistical significance (p-value = 0.9123) so
perhaps a more accurate statement would be that there is no trend
positive or negative between the percentage of attempts taken by a given
player and a the number of wins a team can expect in a given season.

\hypertarget{so-what-can-we-say}{%
\section{So what can we say?}\label{so-what-can-we-say}}

After this analysis, it is safe to say that teams with a well performing
running game can expect on average more wins in a season than a team who
struggles to run the ball. This makes sense as the more rushing yards a
team has, generally the more efficient their offense is and the more
points they will score. All those factors make winning games much
easier.

What we cannot conclude though, is that all those rushing yards need to
go through one person. The percentage of rushing yards and rushing
attempts that go through one person seems to have no significant effect
on the number of wins a team should expect in a season. So in all what
this means is get your rushing yards however you can. Having a running
back by committee would seem to be just as effective as having that one
three down bell cow back that your offense runs through when it comes to
expected added wins. There are other considerations such as:

Does paying a feature back have a negative effect on the other talent a
team can surround him with?

How does having your leading rusher also be your QB effect the running
game and subsequently wins?

What type of back is most effective? i.e.~Hammer vs.~scat
vs.~well-rounded running back

These questions are outside the scope of this analysis but would all be
questions well worth asking.

If you've enjoyed this analysis, follow me on twitter @MrOhDubbs. Feel
free to reach out with questions or with suggestions on what I should do
a statistical dive on next!


\end{document}
